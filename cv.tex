%% start of file `template.tex'.
%% Copyright 2006-2012 Xavier Danaux (xdanaux@gmail.com).
%
% This work may be distributed and/or modified under the
% conditions of the LaTeX Project Public License version 1.3c,
% available at http://www.latex-project.org/lppl/.


\documentclass[12pt,a4paper,sans]{moderncv}   % possible options include font size ('10pt', '11pt' and '12pt'), paper size ('a4paper', 'letterpaper', 'a5paper', 'legalpaper', 'executivepaper' and 'landscape') and font family ('sans' and 'roman')

% moderncv themes
\moderncvstyle{banking}                        % style options are 'casual' (default), 'classic', 'oldstyle' and 'banking'
\moderncvcolor{purple}                          % color options 'blue' (default), 'orange', 'green', 'red', 'purple', 'grey' and 'black'
%\renewcommand{\familydefault}{\sfdefault}    % to set the default font; use '\sfdefault' for the default sans serif font, '\rmdefault' for the default roman one, or any tex font name
%\nopagenumbers{}                             % uncomment to suppress automatic page numbering for CVs longer than one page

% character encoding
%\usepackage[utf8]{inputenc}                  % if you are not using xelatex ou lualatex, replace by the encoding you are using
%\usepackage{CJKutf8}                         % if you need to use CJK to typeset your resume in Chinese, Japanese or Korean

% adjust the page margins
\usepackage[scale=0.79]{geometry}
%\setlength{\hintscolumnwidth}{3cm}           % if you want to change the width of the column with the dates
%\setlength{\maketitlenamewidth}{10cm}        % for the 'classic' style, if you want to force the width allocated to your name and avoid line breaks. be careful though, the length is normally calculated to avoid any overlap with your personal info; use this at your own typographical risks...

% personal data
\firstname{Elsab\'{e}}
\familyname{Ros}
\title{Curriculum Vitae}               % optional, remove the line if not wanted
%\address{Insert}{Address}    % optional, remove the line if not wanted
\mobile{Insert cell number}                     % optional, remove the line if 
%not wanted
%\phone{+2~(345)~678~901}                      % optional, remove the line if not wanted
%\fax{+3~(456)~789~012}                        % optional, remove the line if not wanted
\email{hello@elsabe.dev}                          % optional, remove the line 
%if 
\homepage{elsabe.dev}                    % optional, remove the line if not 
%wanted
%\photo[64pt][0.4pt]{pic.jpg}                  % '64pt' is the height the picture must be resized to, 0.4pt is the thickness of the frame around it (put it to 0pt for no frame) and 'picture' is the name of the picture file; optional, remove the line if not wanted
%\quote{Fight me}                 % optional, remove the line if not wanted

% to show numerical labels in the bibliography (default is to show no labels); only useful if you make citations in your resume
%\makeatletter
%\renewcommand*{\bibliographyitemlabel}{\@biblabel{\arabic{enumiv}}}
%\makeatother

% bibliography with mutiple entries
%\usepackage{multibib}
%\newcites{book,misc}{{Books},{Others}}
%----------------------------------------------------------------------------------
%            content
%----------------------------------------------------------------------------------
\begin{document}
%\begin{CJK*}{UTF8}{gbsn}                     % to typeset your resume in Chinese using CJK
%-----       resume       ---------------------------------------------------------
\makecvtitle

\section{Experience}
\cventry{2017--Current}{Senior Software Developer}{Dariel}{}{}{}%

\subsection{Projects at Dariel}
\cventry{}{2017-Current}{ABSA - PDLC}{}{}{\textit{Web development (Spring)} \newline{} I worked on a project to streamline the creation of new projects. This tool integrates with the Atlassian stack and various other tools to enable project managers to create project spaces and quickly add and remove team members from their projects. The tools also allow developers to set up source code management and CI pipelines using templates. \newline{}
\textit{HTML, CSS, Spring, Java, Maven, Git, Angular, Bamboo, Bitbucket, Confluence, JIRA, TFS, Sonarqube, Nexus} \newline{}}

\noindent\hfil\rule{0.5\textwidth}{.4pt}\hfil\newline

\cventry{2015--2017}{Senior Software Engineer}{Entelect}{}{}{}%
\cventry{2013--2015}{Software Engineer}{Entelect}{}{}{}%

\subsection{Projects at Entelect}
\cventry{}{2015-2017}{Discovery Mobile app}{}{}{\textit{Android development, Backend development (Spring)} \newline{} On this project, we implement new features for the Discovery member app in South Africa, as well as reducing existing technical debt. As part of this project, I also design RESTful microservices and implement them in Spring. Another part I play in this project is supporting and helping junior developers.\newline{} \textit{Android, Gradle, Spring, Spring Boot, RAML, Maven, Sonar, Bamboo, JIRA, Git} \newline{}}

\cventry{}{2013-2015}{DPM Mobile application}{}{}{\textit{Android development} \newline{} I was a member of the team that worked on the mobile application for Discovery Partner Markets. This project involved developing new features, reducing existing technical debt and liasing with developers in other locations to release the app into multiple regions, including Singapore, Australia and Thailand.\newline{} \textit{Android, Maven, Sonar, Jenkins, JIRA, Git} \newline{}}

\cventry{}{2013}{Vitality Service Portal}{}{}{\textit{Web development (Spring)} \newline{} I worked as a member of a team that developed an internal web-based application for use by the client's servicing agents.\newline{}
\textit{HTML, CSS, Javascript, Spring, Java, Maven, Git} \newline{}}

\noindent\hfil\rule{0.5\textwidth}{.4pt}\hfil\newline

\cventry{2012}{Tutor}{University of Pretoria}{}{}{}%
\cventry{2011}{Teaching Assisstant}{University of Pretoria}{}{}{}%

%
%\begin{itemize}%
%\item Achievement 1;
%\item Achievement 2, with sub-achievements:
%  \begin{itemize}%
%  \item Sub-achievement (a);
%  \item Sub-achievement (b), with sub-sub-achievements (don't do this!);
%    \begin{itemize}
%    \item Sub-sub-achievement i;
%    \item Sub-sub-achievement ii;
%    \item Sub-sub-achievement iii;
%    \end{itemize}
%  \item Sub-achievement (c);
%  \end{itemize}
%\item Achievement 3.
%\end{itemize}

%\cventry{year--year}{Job title}{Employer}{City}{}{Description line 1\newline{}Description line 2}
%\subsection{Miscellaneous}
%\cventry{year--year}{Job title}{Employer}{City}{}{Description}

\section{Courses and certifications}
\cventry{November 2016}{Certified Spring Professional}{Pivotal}{}{\textit{}}{}
\cventry{May 2015}{Hardware Security}{Coursera}{}{\textit{}}{Verified Certificate}
\cventry{April 2015}{Cryptography}{Coursera}{}{\textit{}}{Verified Certificate}
\cventry{April 2015}{Software Security}{Coursera}{}{\textit{}}{Verified Certificate}
\cventry{March 2015}{Usable Security}{Coursera}{}{\textit{}}{Verified Certificate}
\cventry{June 2012}{Applied Cryptography}{Udacity}{}{\textit{}}{}
\cventry{December 2011}{Introduction to Artificial Intelligence}{Udacity}{}{\textit{}}{}

\section{Education}
\cventry{2016--}{MSc Computer Science}{University of Pretoria}{}{\textit{}}{Ongoing}

\cventry{2013--2014}{BSc (Hons) Computer Science}{University of Pretoria}{}{\textit{}}{\textit{Dissertation: }An information-gathering botnet for private cloud environments}
	
\cventry{2010--2012}{BIS: Multimedia}{University of Pretoria}{}{\textit{Cum Laude}}{  % arguments 3 to 6 can be left empty
\begin{itemize}%
\item Golden Key Honour Society
\end{itemize}
}

%\section{Master thesis}
%\cvitem{title}{\emph{Title}}
%\cvitem{supervisors}{Supervisors}
%\cvitem{description}{Short thesis abstract}

\section{Languages}
\cvitemwithcomment{English}{Professional}{}
\cvitemwithcomment{Afrikaans}{Home language}{}
\cvitemwithcomment{Dutch}{Elementary}{}

%\section{Computer skills}
%\cvdoubleitem{category 1}{XXX, YYY, ZZZ}{category 4}{XXX, YYY, ZZZ}
%\cvdoubleitem{category 2}{XXX, YYY, ZZZ}{category 5}{XXX, YYY, ZZZ}
%\cvdoubleitem{category 3}{XXX, YYY, ZZZ}{category 6}{XXX, YYY, ZZZ}

%\section{Interests}
%\cvitem{hobby 1}{Description}
%\cvitem{hobby 2}{Description}
%\cvitem{hobby 3}{Description}

%\section{Extra 1}
%\cvlistitem{Item 1}
%\cvlistitem{Item 2}
%\cvlistitem{Item 3}

\renewcommand{\listitemsymbol}{-~}            % change the symbol for lists

%\section{Extra 2}
%\cvlistdoubleitem{Item 1}{Item 4}
%\cvlistdoubleitem{Item 2}{Item 5\cite{book1}}
%\cvlistdoubleitem{Item 3}{}

% Publications from a BibTeX file without multibib\renewcommand*{\bibliographyitemlabel}{\@biblabel{\arabic{enumiv}}}% for BibTeX numerical labels
\nocite{*}
\bibliographystyle{plain}
%\bibliography{publications}                   % 'publications' is the name of a BibTeX file

% Publications from a BibTeX file using the multibib package
%\section{Publications}
%\nocitebook{book1,book2}
%\bibliographystylebook{plain}
%\bibliographybook{publications}              % 'publications' is the name of a BibTeX file
%\nocitemisc{misc1,misc2,misc3}
%\bibliographystylemisc{plain}
%\bibliographymisc{publications}              % 'publications' is the name of a BibTeX file

\clearpage
%-----       letter       ---------------------------------------------------------
% recipient data
\recipient{Company Recruitment team}{Company, Inc.\\123 somestreet\\some city}
\date{January 01, 1984}
\opening{Dear Sir or Madam,}
\closing{Yours faithfully,}
\enclosure{curriculum vit\ae{}}
\makelettertitle

Lorem ipsum dolor sit amet, consectetur adipiscing elit. Duis ullamcorper neque sit amet lectus facilisis sed luctus nisl iaculis. Vivamus at neque arcu, sed tempor quam. Curabitur pharetra tincidunt tincidunt. Morbi volutpat feugiat mauris, quis tempor neque vehicula volutpat. Duis tristique justo vel massa fermentum accumsan. Mauris ante elit, feugiat vestibulum tempor eget, eleifend ac ipsum. Donec scelerisque lobortis ipsum eu vestibulum. Pellentesque vel massa at felis accumsan rhoncus.

Suspendisse commodo, massa eu congue tincidunt, elit mauris pellentesque orci, cursus tempor odio nisl euismod augue. Aliquam adipiscing nibh ut odio sodales et pulvinar tortor laoreet. Mauris a accumsan ligula. Class aptent taciti sociosqu ad litora torquent per conubia nostra, per inceptos himenaeos. Suspendisse vulputate sem vehicula ipsum varius nec tempus dui dapibus. Phasellus et est urna, ut auctor erat. Sed tincidunt odio id odio aliquam mattis. Donec sapien nulla, feugiat eget adipiscing sit amet, lacinia ut dolor. Phasellus tincidunt, leo a fringilla consectetur, felis diam aliquam urna, vitae aliquet lectus orci nec velit. Vivamus dapibus varius blandit.

Duis sit amet magna ante, at sodales diam. Aenean consectetur porta risus et sagittis. Ut interdum, enim varius pellentesque tincidunt, magna libero sodales tortor, ut fermentum nunc metus a ante. Vivamus odio leo, tincidunt eu luctus ut, sollicitudin sit amet metus. Nunc sed orci lectus. Ut sodales magna sed velit volutpat sit amet pulvinar diam venenatis.

\makeletterclosing

%\clearpage\end{CJK*}                         % if you are typesetting your resume in Chinese using CJK; the \clearpage is required for fancyhdr to work correctly with CJK, though it kills the page numbering by making \lastpage undefined
\end{document}


%% end of file `template.tex'.
