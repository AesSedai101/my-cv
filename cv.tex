%% start of file `template.tex'.
%% Copyright 2006-2012 Xavier Danaux (xdanaux@gmail.com).
%
% This work may be distributed and/or modified under the
% conditions of the LaTeX Project Public License version 1.3c,
% available at http://www.latex-project.org/lppl/.


\documentclass[12pt,a4paper,sans]{moderncv}   % possible options include font size ('10pt', '11pt' and '12pt'), paper size ('a4paper', 'letterpaper', 'a5paper', 'legalpaper', 'executivepaper' and 'landscape') and font family ('sans' and 'roman')

% moderncv themes
\moderncvstyle{banking}                        % style options are 'casual' 
%(default), 'classic', 'oldstyle' and 'banking'
\moderncvcolor{purple}                          % color options 'blue' (default), 'orange', 'green', 'red', 'purple', 'grey' and 'black'
%\renewcommand{\familydefault}{\sfdefault}    % to set the default font; use '\sfdefault' for the default sans serif font, '\rmdefault' for the default roman one, or any tex font name
%\nopagenumbers{}                             % uncomment to suppress automatic page numbering for CVs longer than one page

% character encoding
%\usepackage[utf8]{inputenc}                  % if you are not using xelatex ou lualatex, replace by the encoding you are using
%\usepackage{CJKutf8}                         % if you need to use CJK to typeset your resume in Chinese, Japanese or Korean

% adjust the page margins
\usepackage[scale=0.79]{geometry}
%\setlength{\hintscolumnwidth}{3cm}           % if you want to change the width of the column with the dates
%\setlength{\maketitlenamewidth}{10cm}        % for the 'classic' style, if you want to force the width allocated to your name and avoid line breaks. be careful though, the length is normally calculated to avoid any overlap with your personal info; use this at your own typographical risks...

% personal data
\firstname{Elsab\'{e}}
\familyname{Ros}
\title{Curriculum Vitae}               % optional, remove the line if not wanted
%\address{Insert}{Address}    % optional, remove the line if not wanted
\mobile{Insert cell number} 
\email{hello@elsabe.dev} 
\homepage{elsabe.dev}  

% bibliography with mutiple entries
%\usepackage{multibib}
%\newcites{book,misc}{{Books},{Others}}
%----------------------------------------------------------------------------------
%            content
%----------------------------------------------------------------------------------
\begin{document}
%\begin{CJK*}{UTF8}{gbsn}                     % to typeset your resume in Chinese using CJK
%-----       resume       ---------------------------------------------------------
\makecvtitle

\section{Experience}
\cventry{Jan 2019--Current}{Acting Mobile architect}{Discovery 
Holdings}{}{}{Solution design involving mobile applications, investigating new 
tech, improve development process, application development.}
\cventry{Jul 2018--Dec 2018}{Senior Developer}{}{}{}{Android 
development, solution design, improve development process.}

\noindent\hfil\rule{0.5\textwidth}{.4pt}\hfil\newline

\cventry{2017--2018}{Senior Software Developer}{Dariel}{}{}{\textit{}}%

\begin{itemize}
\setlength\itemsep{0.5em}
	
\item[] \textbf{ABSA - IRIS} (\textit{Jun 2018}) \newline B2B (Spring)
\item[] \textbf{ABSA - PDLC }  (\textit{Oct 2017-May 2018}) \newline Web 
development (Spring and Angular), Atlassian APIs
\item[] \textbf{Dariel Graduate Application} (\textit{Apr 2018-Jun 2018}) 
\newline Progressive Web app (Polymer), Firebase

\end{itemize}


\noindent\hfil\rule{0.5\textwidth}{.4pt}\hfil\newline

\cventry{2015--2017}{Senior Software Engineer}{Entelect}{}{}{}%
\cventry{2013--2015}{Software Engineer}{}{}{}{\textit{}}%

\begin{itemize}
\setlength\itemsep{0.5em}

\item[] \textbf{Discovery Mobile application} (\textit{2015-2017}) \newline 
Android development, Backend development (Spring)
\item[] \textbf{DPM Mobile application} (\textit{2013-2015}) \newline Android 
development
\item[] \textbf{Vitality Service Portal} (\textit{2013}) \newline Web 
development (Spring)

\end{itemize}

\noindent\hfil\rule{0.5\textwidth}{.4pt}\hfil\newline

\cventry{2012}{Tutor}{University of Pretoria}{}{}{}%
\cventry{2011}{Teaching Assisstant}{University of Pretoria}{}{}{}%

%
%\begin{itemize}%
%\item Achievement 1;
%\item Achievement 2, with sub-achievements:
%  \begin{itemize}%
%  \item Sub-achievement (a);
%  \item Sub-achievement (b), with sub-sub-achievements (don't do this!);
%    \begin{itemize}
%    \item Sub-sub-achievement i;
%    \item Sub-sub-achievement ii;
%    \item Sub-sub-achievement iii;
%    \end{itemize}
%  \item Sub-achievement (c);
%  \end{itemize}
%\item Achievement 3.
%\end{itemize}

%\cventry{year--year}{Job title}{Employer}{City}{}{Description line 1\newline{}Description line 2}
%\subsection{Miscellaneous}
%\cventry{year--year}{Job title}{Employer}{City}{}{Description}

\section{Formal Education}
\cventry{2016--2018}{MSc Computer Science}{University of 
Pretoria}{}{\textit{}}{\textit{Research: }Digital Forensic Readiness in Mobile 
Device Management Systems}

\cventry{2013--2014}{BSc (Hons) Computer Science}{University of 
Pretoria}{}{\textit{}}{\textit{Research: }An information-gathering botnet for 
private cloud environments}
	
\cventry{2010--2012}{BIS: Multimedia}{University of Pretoria}{}{\textit{Cum 
Laude}}{}

\section{Certifications}
\cventry{November 2016}{Pivotal}{Certified Spring Professional}{}{\textit{}}{}

\section{Courses}
\cventry{May 2015}{Coursera}{Hardware Security}{}{Verified 
Certificate}{}
\cventry{April 2015}{Coursera}{Cryptography}{}{Verified Certificate}{}
\cventry{April 2015}{Coursera}{Software Security}{}{Verified 
Certificate}{}
\cventry{March 2015}{Coursera}{Usable Security}{}{Verified 
	Certificate}{}

\section{Languages}
\cvitemwithcomment{English}{Professional}{}
\cvitemwithcomment{Afrikaans}{Home language}{}


\renewcommand{\listitemsymbol}{-~}            % change the symbol for lists

%\section{Extra 2}
%\cvlistdoubleitem{Item 1}{Item 4}
%\cvlistdoubleitem{Item 2}{Item 5\cite{book1}}
%\cvlistdoubleitem{Item 3}{}

% Publications from a BibTeX file without 
%multibib\renewcommand*{\bibliographyitemlabel}{\@biblabel{\arabic{enumiv}}}% 
%\section{Publications}
%%for BibTeX numerical labels
\nocite{*}
\bibliographystyle{plain}
\bibliography{publications}                   % 'publications' is the name of a 
%BibTeX file

\newpage

\section{Masters thesis}
\cvitem{Title}{\emph{Digital Forensic Readiness in Mobile Device Management 
Systems}}
\cvitem{Supervisor}{Professor HS Venter}
\cvitem{Description}{Mobile devices have become very popular, and virtually 
everyone owns a smart device. As more employees became owners of smart devices, 
the organisations were put under pressure to allow employees to use their smart 
devices for work purposes, or alternatively provide employees with smart 
devices. \newline \newline
Most organisations opted for a Bring Your Own Device policy, where 
employees use their own smart devices for work purposes, with the 
organisation reimbursing some of the costs. Adopting such a policy 
introduced risks into the organisations, since the organisations do not own 
and do not have direct control over employees' personal devices. \newline 
\newline
One of the most widely used solutions to this problem is Mobile Device 
Management (MDM) software, which is installed on employees' devices and 
prevent them from taking actions that may be harmful to the 
organisation.\newline \newline
This leads us to the problem statement of this research. Since MDM systems 
are purely preventative and devices are not owned by the organisation, it 
is expensive and sometimes impossible for organisations to retrieve 
potential evidence from the devices when an incident occurs.\newline \newline
This research proposes a model to solve this problem by introducing a 
digital forensic readiness component into an MDM system. Adding digital 
forensic readiness to an existing MDM solution reduces costs by collecting 
evidence when suspicious activity is detected, reducing investigation times 
and legal costs involved in collecting evidence.\newline \newline
A prototype was created to show that the proposed model could be 
implemented in practice. The prototype shows how this solution can be 
utilised to collect data from devices and utilise it in an 
investigation.\newline \newline
Finally, the research and prototype are critically evaluated, and the 
benefits and shortcomings of such a solution are presented. The author also 
addresses privacy concerns arising from the data collection component.}

% Publications from a BibTeX file using the multibib package
%\section{Publications}
%\nocitebook{book1,book2}
%\bibliographystylebook{plain}
%\bibliographybook{publications}              % 'publications' is the name of a BibTeX file
%\nocitemisc{misc1,misc2,misc3}
%\bibliographystylemisc{plain}
%\bibliographymisc{publications}              % 'publications' is the name of a BibTeX file

%\section{Computer skills}
%\cvdoubleitem{category 1}{XXX, YYY, ZZZ}{category 4}{XXX, YYY, ZZZ}
%\cvdoubleitem{category 2}{XXX, YYY, ZZZ}{category 5}{XXX, YYY, ZZZ}
%\cvdoubleitem{category 3}{XXX, YYY, ZZZ}{category 6}{XXX, YYY, ZZZ}

%\section{Interests}
%\cvitem{hobby 1}{Description}
%\cvitem{hobby 2}{Description}
%\cvitem{hobby 3}{Description}

%\section{Extra 1}
%\cvlistitem{Item 1}
%\cvlistitem{Item 2}
%\cvlistitem{Item 3}

%\section{Master thesis}
%\cvitem{title}{\emph{Title}}
%\cvitem{supervisors}{Supervisors}
%\cvitem{description}{Short thesis abstract}

%\clearpage
%-----       letter       ---------------------------------------------------------
% recipient data
%\recipient{Company Recruitment team}{Company, Inc.\\123 somestreet\\some city}
%\date{January 01, 1984}
%\opening{Dear Sir or Madam,}
%\closing{Yours faithfully,}
%\enclosure{curriculum vit\ae{}}
%\makelettertitle

%Lorem ipsum dolor sit amet, consectetur adipiscing elit. Duis ullamcorper 
%neque sit amet lectus facilisis sed luctus nisl iaculis. Vivamus at neque 
%arcu, 
%sed tempor quam. Curabitur pharetra tincidunt tincidunt. Morbi volutpat 
%feugiat 
%mauris, quis tempor neque vehicula volutpat. Duis tristique justo vel massa 
%fermentum accumsan. Mauris ante elit, feugiat vestibulum tempor eget, eleifend 
%ac ipsum. Donec scelerisque lobortis ipsum eu vestibulum. Pellentesque vel 
%massa at felis accumsan rhoncus.

%Suspendisse commodo, massa eu congue tincidunt, elit mauris pellentesque orci, 
%cursus tempor odio nisl euismod augue. Aliquam adipiscing nibh ut odio sodales 
%et pulvinar tortor laoreet. Mauris a accumsan ligula. Class aptent taciti 
%sociosqu ad litora torquent per conubia nostra, per inceptos himenaeos. 
%Suspendisse vulputate sem vehicula ipsum varius nec tempus dui dapibus. 
%Phasellus et est urna, ut auctor erat. Sed tincidunt odio id odio aliquam 
%mattis. Donec sapien nulla, feugiat eget adipiscing sit amet, lacinia ut 
%dolor. 
%Phasellus tincidunt, leo a fringilla consectetur, felis diam aliquam urna, 
%vitae aliquet lectus orci nec velit. Vivamus dapibus varius blandit.

%sagittis. Ut interdum, enim varius pellentesque tincidunt, magna libero 
%sodales tortor, ut fermentum nunc metus a ante. Vivamus odio leo, tincidunt eu 
%luctus ut, sollicitudin sit amet metus. Nunc sed orci lectus. Ut sodales magna 
%sed velit volutpat sit amet pulvinar diam venenatis.

%\makeletterclosing

%\clearpage\end{CJK*}                         % if you are typesetting your resume in Chinese using CJK; the \clearpage is required for fancyhdr to work correctly with CJK, though it kills the page numbering by making \lastpage undefined
\end{document}


%% end of file `template.tex'.
